\documentclass{article}
\sloppy
\title{The US and Rampant Individualism }
\author{Pierson Lipschultz}
\usepackage{biblatex} 
\addbibresource{sources.bib}

\begin{document}
\maketitle



\section{Introduction}
    The United States has had, ever since its very inception, a deep problem of individuals having a much greater influence then they should politically, both at home and abroad. We have seen this is an enormous amount of conflicts and decisions, but some of the most key examples of this are the Bay of Pigs, and more recently, the war in Iraq. \footnote{I feel like it is very important to make a clear distinction here. As I did research for this paper, a recurrent  theme I found was recently appointed government officials talking about \textit{"waste"} in USAID, but their claims very quickly fall apart when placed under scrutiny. A key example of this being the "\$20 million for Sesame Street", which does not hold up.}

    \cite{CWC_2011}

\section{Bay of Pigs}
    The Bay of Pigs is undoubtedly one of the most public and embarrassing moments in the history of the CIA. It's a disaster so large people look at it and truly wonder how it could have even happened, it's all because of the individualistic nature of policy, both within the organization and broader government. This ideology is evident in the actions of key figures such as Richard Bissell and John F. Kennedy, whose personal motives and approaches played a pivotal role in the planning and execution of said invasion. Bissell's individualism stemmed from a desire for power and respect, while Kennedy's revolved around control. Examples of this dynamic are also present in the actions of high-ranking CIA officials and Nixon's influence.

    Richard Bissell became the CIA's Director of Plans on January 1, 1959, and just over a year later, he proposed a plan to overthrow Fidel Castro to President Eisenhower. This bold move by someone so recently appointed raises the question: what motivated it? The answer lies in Bissell's individualistic goals. Simply put, he wanted to “go big or go home.” As Peter Wyden highlights in Bay of Pigs: The Untold Story, “His attitude was a career gamble… If the operation succeeded, Bissell would be the unquestioned hero of the agencies' most ambitious success.” Bissell sought to secure a legacy and gain honor and respect, viewing Castro's removal as his path to these goals, and to do this he took an approach which was all or nothing.

    Bissell's individualism is additionally evident in his decision-making process. For instance, he coordinated an offer to the Mafia, promising \$150,000 for Castro's assassination—a decision made by him and one other person.  Allen Dulles, the CIA Director at the time, was not briefed on the assassination bribe until over six months later \cite{Wyden1979}. This highlights the fragmented nature of the CIA, which often operated as isolated factions rather than a unified organization. Hiring two members of the Mafia, listed among the FBI's “top ten most-wanted criminals,” underscores this. According to Wyden, “Bissell knew who he was dealing with… On October 18th, he received a memo from the FBI.” Such actions exemplify the rivalry between government agencies, as CIA decisions were reportedly influenced by “bureaucratic rivalry and the relative prestige of rival intelligence organizations, notably the military intelligence agencies and the FBI.” \cite{JeffreysJones2003}. In this case, the CIA actively worked with individuals the FBI was trying to apprehend—a clear conflict of interest and an example of a splintered government.

\section{Commission on Wartime Contracting (CWC) Report}
    In 2008 congress created the independent and bipartisan Commission on Wartime Contracting in Iraq and Afghanistan \cite{CWC_2011}. The CWC was founded in order to find places with excessive waste and fraud and to provide recommendations to congress on how to improve. This commission provided five reports, however, the one detailed here is the final of them. This report is one of the better documentations of excess spending and waste in Iraq. They CWC found that at least \$31 billion, with a possibility of up to \$60 billion, was lost to waste a and fraud. \footnote{It is important to note that this commission will, of course, have a large amount of bias, and this estimate is most likely an underestimation} 

\section{Trump, Elon, and friends}
\pagebreak
\printbibliography[
    heading=bibintoc,
    title={\centering Sources}
    ]

    
\end{document}