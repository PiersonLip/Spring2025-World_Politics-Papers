% why is a negotiated settlement unlikely 
% - the conflict is currently active, "Forcing a settlement under these conditions is likely to backfire, with far-reaching consequences for international security."

% -"The idea that a compromise can be struck with Putin is a dangerous fantasy... There is no division of Ukrainian territory or set of security policies that can address the underlying causes of the war and foster a sustainable peace."

% how is this ethno nationalist conflict?

%  - Leading figures in Putin’s inner circle regularly call Ukrainians (once supposed to be in no way different from Russians) dehumanizing names such as “cockroaches,” “lice,” “rabid mongrels . . . choking on their toxic saliva,” a “hated Nazi tumor,” “nonhumans,” “evil spirits,” and “bloody possessed satanists” in need of extermination. An ex-leader of the Russian-backed Donetsk People’s Republic told Ukrainians via Russian state television: “We will kill as many of you as we have to. We will kill one million, or five million; we can exterminate all of you until you understand that you’re possessed and you have to be cured.”

% - active warcrimes and genocide

% - "The April 2022 discoveries in Bucha, Hostomel, Irpin, and Mariupol of mass graves of Ukrainian civilians who had been bound, tortured, and executed was a sign that Russia’s anger at early failures was being redirected against noncombatants singled out for their nationality."

% - twenty-thousand Ukrainian children have been forcibly abducted and sent to orphanages, foster homes, or adoptive families in Russia. There they are subjected to harsh indoctrination and reeducation programs that seek to erase their Ukrainian identity. 

% - Lest anyone doubt the sincerity of Putin’s campaign to erase future generations of Ukrainians, one only need to look at the deadly bombings of Ukrainian maternity and children’s hospitals throughout the war, far from the front lines and any military targets.

% -Whatever Moscow’s original aims, the conflict is now an ethnonational struggle, a war of peoples. 

% -It would not be easy for Russia to walk away from the ethnic war it has engineered, either. Putin understands a truth known to generations of statesmen before him: Nationalism is a powerful motivator in war and the lynchpin of the mass army in modern warfare.11 As economic conditions worsen in Russia and more Russian soldiers come home maimed and disabled (if they return at all), Putin becomes ever more dependent on a radically nationalistic ideology to fill the ranks, justify his war, and legitimize his rule

% -No wonder that civil wars in ethnically polarized societies have been found to last longer than less polarized societies: Wars over identity, once unleashed, are nearly impossible to return to Pandora’s box.12 Each side has spilled too much blood in the name of identity to easily walk back from the brink, let alone negotiate a just and durable peace between societies infused with nationalist animus for the other. No peace treaty — not least one forced on Ukraine under international pressure — can induce compromise when a people’s existence hangs in the balance.




\documentclass{article}
\usepackage[backend=biber, style=mla]{biblatex} 
\usepackage{setspace}
\doublespacing
\sloppy
\title{Should Ukraine Negotiate with Russia?}
\author{Pierson Lipschultz}
\doublespace
\addbibresource{sources.bib}
\begin{document}
\maketitle

\section{Negotiation Prospects}
% What is Person’s core claim abut prospects for a negotiated settlement to the conflict?
    Person's claim is that a negotiated settlement is incredibly unlikely. He says that despite the notation that most wars end in negotiation, the war in Ukraine is the exception to that. Due to the nature of the war, a negotiated settlement would only serve as a temporary damper, not actually preventing the war.

    \begin{quote} \label{}
       ``The implication is that it is time for the West to put pressure on Kyiv to come to the bargaining table and make concessions for peace\ldots Such views are misguided, misinformed, and downright dangerous: They reveal a fundamental misunderstanding of what Russia and Ukraine are fighting over \ldots [which are] the underlying reasons why Russia launched its invasion\ldots. Make talk of a negotiated end to the war wishful thinking.'' \parencite{person_2025}
    \end{quote}

    Additionally, due to the current state of the conflict, Person says that a settlement is incredibly unlikely, with him stating that:

    \begin{quote}
        ``Forcing a settlement under these conditions is likely to backfire, with far-reaching consequences for international security.'' \parencite{person_2025}
    \end{quote}

\section{Negotiation Obstacles}
% According to the bargaining model discussed in the article, what are the three obstacles to a negotiated settlement?

    There are three main obstacles that prevent the bargaining model from working in the Ukraine conflict. Person states that 
    \begin{quote}
        ``The bargaining model\ldots also identifies three things that can prevent the finding of a mutually preferable negotiated settlement: information problems, issue indivisibility, and commitment problems.'' ~\parencite{person_2025}
    \end{quote}

    \subsection{Information Problems}
        A negotiation of any type requires a mutual agreement of the current state of the war. This means that both sides understand that it is inevitable that one side will win or that the conflict will inevitably end in a draw. Person states that

        \begin{quote} 
            ``Leaders will prefer to carry on fighting as long as they think they have a chance of winning\ldots. Only when both sides implicitly agree about who will win future rounds of combat do they turn to the negotiating table as the losing side foresees its defeat and seeks terms. Alternatively, if both sides are exhausted and agree that a permanent stalemate has set in, they may eventually be willing to negotiate a settlement.'' \parencite{person_2025}
        \end{quote}

        This is incredibly prevalent in the Ukraine conflict due to Putin's power hierarchy, as because he is a dictator, no one will tell him that a conflict or battle \textit{is not} in his favor. This is a key reason why Putin invaded Ukraine in the first place, as he believed that the Ukrainians wouldn't resist, or in some cases, even welcome the Russians. Of course, this turned out to be completely incorrect. As person said: 
        
        \begin{quote}
            ``The problems afflicting his private information were made public by the act of war, but not before this flawed information had made him unwilling to negotiate beforehand. He had believed he was going to win easily.''\parencite{person_2025}
        \end{quote}
        
        This complete asymmetry of information will prevent Putin from ever realizing that the conflict itself is lost or a draw, thus never coming to the table. Especially due to the fact that the power to end the war rests solely in Putin's hands, with Person stating that 

        \begin{quote}
           ``Putin alone decided to start the war, and any decision to end it will be his alone as well. \ldots He does not have advisors who bring him competing options to weigh. There is no exchange of ideas at the top, and there are no quasi-independent power centers \ldots that can press him one way or another.'' \parencite{person_2025}
        \end{quote}
    
    \subsection{Issue Indivisibility}

        Additionally, the war of Ukraine is not an issue which both sides can come to a simple agreement and split everything in half, it is instead an ``all or nothing'' approach. Person states:
        
        \begin{quote}
            ``Some questions are “all or nothing.” They cannot be bargained. While there are in theory very few things that are truly indivisible, many things over which states fight — including identity, ideas, and sovereignty — are extraordinarily difficult to divide between winner and loser. \parencite{person_2025}''
        \end{quote}
        
        This war is a matter of Ukrainians sovereignty, an issue which cannot have a line drawn in the middle. The end that Putin sees, and the reason which he started the conflict, was in order to strip Ukraine of its independence. 

    \subsection{Commitment Problems}

        Because of the depth and animosity involved in the invasion of Ukraine it is unlikely that any long-term settlements will remain unbroken. Person states that:
        \begin{quote}
            ``Even when a negotiated settlement is acceptable to both sides, there are often few ways to credibly demonstrate that each side will uphold their bargain in the future. Adversaries with a long history of conflict may not trust each other’s promises to lay down arms permanently.~\parencite{person_2025}''
        \end{quote}

        This is largely due to the amount of distrust that is spurred on by the conflict, especially in cases where the conflict is ethnonationalist in nature, as is Ukraine.

        \begin{quote}
           ``Russian war aims escalated from destruction of Ukrainian democracy to the physical liquidation of the Ukrainian people\ldots Lest anyone doubt the sincerity of Putin’s campaign to erase future generations of Ukrainians, one only need to look at the deadly bombings of Ukrainian maternity and children’s hospitals throughout the war\ldots. Whatever Moscow’s original aims, the conflict is now an ethnonational struggle, a war of peoples. ''
        \end{quote}

        This ethnonationalist conflict will of course lead to deep-rooted racism and distrust between the two countries, an issue that cannot be simply fixed. This distrust, as previously stated, will prevent any negotiations from being long-lasting, as person states that

        \begin{quote}
            ``[w]ars over identity, once unleashed, are nearly impossible to return to Pandora’s box.''
        \end{quote}

\section{Putin's Impetus}
    % The author argues NATO expansion is not Putin’s central concern as the impetus for the conflict. What is the central reason and why does that make a negotiated settlement difficult?

    Putin's public reasoning is for the invasion of Ukraine because he wishes to prevent the expansion of NATO territory against Russian borders. However, as Person is quick to point out, this argument quickly falls away under scrutiny. As Person says,

    \begin{quote}
        ``It is not NATO membership or territory but Ukrainian democracy — and its fundamental incompatibility with Russian autocracy — that is at the heart of Putin's decision to go to war.'' \parencite{person_2025}
    \end{quote}

    Person goes on to state that Russia's conflict with Ukraine did not start with the invasion, instead starting with Russia's attempts to meddle in the Ukrainian elections. Once it became clear that it would be possible stamp out Ukrainian democracy subtly, turning Ukraine in to a puppet state akin to Belarus, Putin turned to force. Putin is determined to undermine Ukrainian democracy, as according to Person

    \begin{quote}
        ``In Putin's mind Ukrainian democracy poses an existential threat to his power and the survival of his regime in Russia.'' \parencite{person_2025}
    \end{quote}

    A large part of this is due to Putin's personal experience during the Cold War. As a KGB officer during it, Putin's current methodology and thought process is fully based on that experience. In conclusion 

    \begin{quote}
        ``These democratic openings, not NATO expansion, are the true source of Putin's hostility to Ukraine and the democratic West.'' \parencite{person_2025}
    \end{quote}

\section{What Can be Done?}
    % Evaluate the four possible outcomes in the section, “What is to be done?” Which do you find compelling and why?
    
    \subsection{A Western Withdrawal}
        In this scenario the US would remove all support for Ukraine, in which case they would quickly fall to Russian forces. This is, of course, what Putin hopes will happen. This would cripple the idea of the US as a beacon for democracy, allowing a country to fall to a cruel dictatorship. While this seems incredibly unlikely, with every passing day the chance grows. Hence, why there is a rapid militarization currently occurring in Europe.

    \subsection{War of Attrition}
        % 2) A prolonged war of attrition that Ukraine loses
        This is what is currently happening and what I believe to be the most likely outcome. The current supplied amount of aid is just enough to keep Ukraine afloat, however, not enough to actually prevent Russian aggression. This means that Ukraine is slowly being whittled down, losing more and more territory. Person states that
        
        \begin{quote}
            ``Ukraine is closer to exhaustion than Russia is, owing to the one thing that the West would probably never send: manpower. Short of one-sided and transformative technological advances, Kyiv fares worse under a stalemate than Moscow. Putin knows this and is willing to slowly bleed Ukraine dry, even as his armies struggle to make gains.'' \parencite{person_2025}
        \end{quote}

        I believe this is the most likely outcome because of the nature of US bureaucracy. US support is not decisive enough to actually pass the large amount of aid which is needed. We have seen this before where large aid packages were stalled and argued over for so long that it had serious repercussions on the front itself.

    \subsection{Just Enough}
        % 3) The West sends enough help that Ukraine can defend itself while making Russia’s long-run costs unbearable

        In this scenario the US would up the quantity of aid, especially in terms of air munitions, to the up that point that Russia economically could no longer afford to continue funding the war. These munitions would be heavily focused on both long range missiles, allowing Ukraine to strike deeper into Russia \footnote{Long range missiles and defense are currently being supplied in much greater quantity then when this article was written.} This would fatigue Russia, as they current level of aggression is not sustainable. Additionally, modern Western air defense systems are needed, as they would allow for Ukraine to defend against current Russian air attacks, as well as inevitable future ones. Person states that 
        
        \begin{quote}
            ``Eventually, strengthened Ukrainian defenses combined with the compounding costs of offensive operations for both sides could lead to a battlefield stalemate.\ldots Russia would keep trying air attacks, so air defense would remain an active sector for Ukraine. Even if the shooting simmers down, Putin's maximalism suggests that he would carry on his hybrid war against Ukraine indefinitely.'' \parencite{person_2025}
        \end{quote}

        Hence, the critical need for long term air defense. Person says this would be easy to put in a neighboring NATO country like Romania or Poland and would allow coverage over a large section of Ukraine.
        
        While this plan does lead to an eventual Western `victory,' it never conclusively ends the war in Ukraine, with Person stating that 

        \begin{quote}
            ``This is the Korean solution: A half-century from now, Ukraine might be living with a Russian border that looks a lot like the Korean Demilitarized Zone does today (except far longer): It would be heavily fortified and stable, with bitter enemies glaring at each other across its razor wire, tank traps, and trenches.'' \parencite{person_2025}
        \end{quote}

        This is, of course, not ideal. 

    \subsection{Whatever it Takes}
        This is by far the best outcome, however, it is incredibly unlikely. In this scenario the US does whatever it takes to ensure a Ukrainian victory. This would conclusively show Putin that victory in Ukraine is impossible, leading to a Russian withdrawal. This outcome has numerous upsides, including 
            \begin{itemize}
                \item Reclamation of currently Russian occupied territory 
                \item Show power of democracy
                \item Prevent further acts of Russian aggression 
            \end{itemize}
        This outcome has, by far, the best long term outcomes, by fully supporting Ukraine Person states that 

        \begin{quote}
            ``Securing long-term peace, prosperity, and democracy in Europe will require equally bold action if we hope to restore a Europe that is whole, free, and at peace. Only by making Ukraine a full member of NATO and the EU will we ensure that Russia can never make war against Ukraine again.''
        \end{quote}

        Person mentions that that large part of the Western argument against this option is the threat of nuclear escalation. He very rapidly dismisses this point without much of an actual discussion of it, yet I do agree that it is not as large of a threat as the West makes it out to be. Russians nuclear armament policy is shockingly robust, having a lot of fail-safe measures. I do fully agree with Person when he says that 

        \begin{quote}
            ``[W]e should not allow what remains a low-probability scenario to paralyze decisive action in support of Ukraine.'' \parencite{person_2025}
        \end{quote}
        
    \subsection{Where to go}
        I believe that full Western support of Ukraine is absolutely a necessity, despite the political unlikeliness. This is because it provides a concrete long-term solution, preventing Ukraine from falling to a cruel government hell-bent on their extermination. Through the conflict in Ukraine we have clearly seen that Russia disregards human rights violations nor are they committed to long term peace agreements. I believe that unless there is full Western support which decisively ends the conflict in Ukraine's favor, this conflict can only end in a way that is a permanent detriment to Ukrainian sovereignty, regional stability, and world democracy.
\end{document}