\documentclass{article}
\usepackage[backend=bibtexm, style = mla]{biblatex}
\usepackage{setspace}
\sloppy
\title{The US and Rampant Individualism }
\author{Pierson Lipschultz}

\doublespace
\addbibresource{sources.bib}

\begin{document}
\maketitle

% indenfity if it is unkown quote 
% why these cases
    % national security archvie
    % mla or apa
% redployment first
\begin{abstract}
    In this paper I investigate how certain US political figures and their personal decisions have effected global politics.  
\end{abstract}


\section{Introduction}
    The United States has had, ever since its very inception, a deep problem of individuals having a much greater influence then they should politically, both at home and abroad. We have seen this is an enormous amount of conflicts and decisions, but some of the most key examples of this are the Bay of Pigs, and more recently, the war in Iraq.~\footnote{I feel like it is very important to make a clear distinction here. As I did research for this paper, a recurrent  theme I found was recently appointed government officials talking about \textit{``waste''} in USAID, but their claims very quickly fall apart when placed under scrutiny. A key example of this being the ``\$20 million for Sesame Street'', which does not hold up.}

\section{Bay of Pigs}

    The Bay of Pigs is undoubtedly one of the most public and embarrassing moments in the history of the CIA.\@ It's a disaster so large people look at it and wonder how it could have even happened, but in actuality is quite simple. It was due to the individualistic nature of policy, both within the organization and broader government. This ideology is evident in the actions of key figures such as Richard Bissell and John F. Kennedy, whose personal motives and approaches played a pivotal role in the planning and execution of said invasion. Bissell's individualism stemmed from a desire for power and respect, while Kennedy's revolved around control. Examples of this dynamic are also present in the actions of high-ranking CIA officials and Nixon's influence.~\cite{JeffreysJones2003}

    Richard Bissell became the CIA's Director of Plans on January 1, 1959, and just over a year later, he proposed a plan to overthrow Fidel Castro to President Eisenhower~\cite{Wyden1979}. This bold move by someone so recently appointed raises the question: what motivated it? The answer lies in Bissell's individualistic goals. Simply put, he wanted to “go big or go home.” As Peter Wyden highlights in \textit{Bay of Pigs: The Untold Story}, 
    \begin{quotation}
        “His attitude was a career gamble… If the operation succeeded, Bissell would be the unquestioned hero of the agencies' most ambitious success.”
    \end{quotation} 
    Bissell sought to secure a legacy of his own, gaining honor and respect in the process, and he viewed Castro's removal as his path to this, taking an approach which was all or nothing.

    Bissell's individualism is additionally evident in his decision-making process. For instance, he coordinated an offer to the Mafia, promising \$150,000 for Castro's assassination—a decision made by him and one other person~\cite{Wyden1979}. Allen Dulles, the CIA Director at the time, was not briefed on the assassination bribe until over six months later~\cite{Wyden1979}. This highlights the fragmented nature of the CIA, which often operated as isolated factions rather than a unified organization. Hiring two members of the Mafia, listed among the FBI's “top ten most-wanted criminals,” underscores this. According to Wyden, \begin{quotation}
        “Bissell knew who he was dealing with… On October 18th, he received a memo from the FBI.”~\cite{Wyden1979}
    \end{quotation}  
    However, despite knowing that the FBI was actively pursuing the mobsters, he still chose to hire them. Actions like these exemplify the rivalry between government agencies, as CIA decisions were reportedly influenced by 
    \begin{quotation} 
        “bureaucratic rivalry and the relative prestige of rival intelligence organizations, notably the military intelligence agencies and the FBI.”~\cite{JeffreysJones2003}.
    \end{quotation} 
    In this case, the CIA actively worked with individuals the FBI was trying to apprehend—a clear conflict of interest and an example of a splintered government.

    \textit{Note, I plan to add a section here tying how this individualistic policy affected Cuba itself, how events (like the bay of pigs) created more unrest and instability within Cuba. I.e. \textbf{what the effects this policy on Cuba was}.}
\section{Iraq}
    \subsection{Commission on Wartime Contracting (CWC) Report}
        In 2008 congress created the independent and bipartisan Commission on Wartime Contracting in Iraq and Afghanistan~\cite{CWC_2011}. The CWC was founded in order to find places with excessive waste and fraud and to provide recommendations to congress on how to improve. This commission provided five reports, however, the one detailed here is the final of them. This report is one of the better documentations of excess spending and waste in Iraq. They CWC found that at least \$31 billion, with a possibility of up to \$60 billion, was lost to waste a and fraud.\footnote{It is important to note that this commission will, of course, have a large amount of bias, and this estimate is most likely an underestimation} The waste was in a couple of key sectors, which the largest one being overzealous contracts with PMC groups and fraud within them, but also, to a lesser extent, within USAID programs.

        However, it is not just taxpayer dollars which are lost, the CWC argues that lives were also lost. 

        The CWC was incredibly critical of the government, exposing just how truly rampant the waste was. During the final hearing of the CWC, Claire McCaskill said that 
        \begin{quote}
            "I have taken trips to Iraq and Afghanistan, where I have seen with my own eyes the lack of planning, inadequate oversight, and sheer waste in our contingency contracting operations. I can tell a number of anecdotal stories about my visits to both Iraq and Afghanistan on contracting oversight trips. But I particularly remember the time when I asked a general in Kuwait, where a lot of the contracting work was done, 'how did this happen? How did this get so out of control?'~\cite{us_senate2011wartime_contracting}"
        \end{quote}
        This is after a general said to her that 
        \begin{quote}
            "I wanted three kinds of ice cream in the mess hall yesterday, and I didn't care what it cost."~\cite{us_senate2011wartime_contracting}
        \end{quote}
        The issue here is \textit{not} the spending, it's the fact that one general had the individual power to do that. 
        When the witness senator Kelly Ayotte was called up to the stand, she stated that \begin{quote}
            "If\ldots we spend large quantities of international contracting funds quickly and with insufficient oversight, it is likely  that some of those funds will unintentionally fuel corruption, finance insurgent organizations, strengthen criminal patronage networks, and undermine our efforts in Afghanistan." \cite{us_senate2011wartime_contracting}
        \end{quote}
        This is \textit{exactly} what happened, as stated by the BBC, it was found that Hazem Shalaan, the Iraqi prime minister of defense, did just that. The BBC state that
        \begin{quote}
            "He and his associates siphoned an estimated \$1.2bn out of the ministry. They bought old military equipment from Poland but claimed for top-class weapons. Meanwhile[,] they diverted money into their own accounts."
        \end{quote}
        This goes to show that misscordinated funding has larger repercussions than just lost taxpayer dollars, it in fact, undermines the very goal of US's involvement in Iraq and leads to lasting negative consequences. 

        \textit{Note: I plan to add a lot more here regarding the CWC report and the hearing, however, due to the very lengthy nature of both, I have not had the time to fully comb through them yet. }

    \subsection{Redeployment}
        \begingroup     
            \centering
            \textit{\textbf{Note:} I am on the edge with whether this is needed, both the CWC report and the Halliburton situation are very convincing, and the inclusion of a non-fiction story might serve to weaken the argument as a whole, as it serves as more of a fictional anecdote then anything else. Love to hear your thoughts on this. If I were to include it I could write a lot more about it, with direct quotes as such, just didn't want to do that just to scrap it. }
        \endgroup\\[15pt]
        In \textit{Redeployment} by Phil Klay \cite{Klay2014}, a \textbf{fictional}\footnote{\textbf{Disclaimer:} While this is a fictional story, it is written by a veteran, which I think gives it a relative amount of merit. Unfortunately, it is incredibly hard to find actual specific accounts from Iraq, most likely due to the recent nature of it and the lack of declassified files. That being said, as it is fiction, most likely is quite hyperbolic} story written a veteran from Iraq, we see a collection of short stories based off of his deployment one. One of them, \textit{Money as a Weapons System} describes someone's experience working for USAID in Iraq. He finds that he is part of frivolous programs and is forced to  go through various contracts due to senators or other government officials who do not truly know what is going on the ground. 

        Our main character finds that certain people have absolutely overkill wages, that programs are being run with no forcible conclusion, and that he is forced to impose the will of senators who have true grasp of what's going on. 

    \subsection{Halliburton, Dick Cheney, and a hell of a lot of oil}
        In Iraq a lot of government officials had direct and key ties to oil companies, with politicians receiving large amount of funding from these companies for certain policy decisions

        Rampant corruption is overpresent in government, but in times of war, it becomes even more prevalent. Corruption is not the focus here, it is how \textit{one} person's corruption, Cheney's, impacted Iraq. Cheney's love for oil in Iraq is a great example of this.

        From 1995 to 2000 Dick Cheney was the CEO of Halliburton, a large oil company. Before this he was the secretary of defense from 1989 to 1993 and afterward was Vice President from 2001 to 2009. This tie already would raise some eyebrows, but it quickly becomes incredibly egregious. 
        
        Halliburton had a very involved role in Iraq, as they won a government contract. This contract was incredibly lucrative, drawing in an estimated~\$10 billion. A very important aspect is this contract was \textit{non-competitive}, only Halliburton got to bid on it \cite{bbc2008iraqcontracts}. This is very clearly some sort of internal favoritism, and with the former CEO as VP,\footnote{Import disclaimer here. As with more recent stories, there are a lot of sources from both the left and the right saying widely different things (Cheney is still on the payroll, Cheney 
        isn't, Cheney's still making money off of his stocks, Cheney's stocks are going to charity, etc etc. The truth is, of course, somewhere in the middle.)}
        it seems clear where that comes from. The chief overseer of contracts at the Army Corps of Engineers, Bunnatine H. Greenhouse, was fired from her position after saying that the contract was 
        \begin{quote}
            "[T]he most blatant and improper contract abuse I have witnessed during the course of my professional career. \cite{nytimes2005halliburtoncontract}"
        \end{quote}  
        Contracts like these are a key aspect of waste in government. Greenhouse states that they discovered Halliburton to be actively charging artificially higher rates for gas to soldiers. \cite{nytimes2005halliburtoncontract}

        \textit{Note: This feels mostly done, needs some polishing, and will probably add another example and some more quotes, however, there is not a lot of \textbf{unbiased} coverage. }
        
        

\section{Trump, Elon, and friends}
    There is a very good argument to be made that this idea of individualism in US policy is more prevalent than it ever has been before. The Trump administration is full of individual actors, all of who have an incredibly large influence on both the US and the world as a whole. We have seen this with a number of actors, including 
    \begin{itemize}
        \item RFK
        \item Hegseth 
        \item Trump
        \item Elon
    \end{itemize}

    \textit{\textbf{Note:} I realized while writing this it would be very possible to directly tie the interests of certain individuals in the current political scene with this sense of individualistic policy. It would be a very recent tie in. I was planning on looking into Elon and RFK primarily, as Elon has been public about his ideas of what government should be, talking about neorealism in interviews and such. His very personal ideas have a large effect on global politics. I am debating talking about Trump himself as well.}
\pagebreak
\printbibliography[
    heading=bibintoc,
    title={\centering Sources}
    ]

    
\end{document}