\documentclass[12pt, twoside]{article}
\usepackage[
  backend=biber,
  sorting=none,
  style=mla
]{biblatex}
\usepackage{setspace}
\usepackage{hyperref}


\sloppy
\title{The U.S. and Rampant Individualism }
\author{Pierson Lipschultz}

\doublespace
\addbibresource{sources.bib}

\begin{document}
\maketitle

% indenfity if it is unkown quote 
% why these cases
    % national security archvie
    % mla or apa
% redployment first
\begin{abstract}
    In this paper I investigate the question ``How have certain U.S. political figures and their personal agendas affected world politics?'' I investigate this by using evidence from the Bay of Pigs invasion as well as the more recent war in Iraq. I find and name certain U.S. officials who were key players in both conflicts.
\end{abstract}

\section{Introduction}

    The United States has, since its inception, had a deep problem with individuals having a much greater influence then they should politically, both at home and abroad. These individuals have had a large impact on global issues, directly contributing to the current state of the world. We have seen this is in both U.S. conflicts and decisions and some of the most prominent examples is the Bay of Pigs invasion and the more recent war in Iraq. I chose to investigate the Bay of Pigs in particular, as, due to its older nature, a vast majority of the records associated with it have been declassified, allowing for an in-depth analysis. I corroborated this event with the war in Iraq, as provided a glimpse into how this issue still remains in recent times. Furthermore, I chose to avoid very current events, as while the current administration could be a clear example of this, the lasting consequences are not yet known and current coverage is heavily politicized. Lastly, I chose to focus on just the U.S. as it prevented the scope of the project from becoming to broad, however, this is absolutely an issue in other countries.

\section{Bay of Pigs}

    The Bay of Pigs is undoubtedly one of the most public and embarrassing moments in the history of the CIA.\@ This disaster was due to the individualistic nature of policy within the CIA, FBI, and government as a whole. Such ideology is evident in the actions of key figures such as Richard Bissell and John F. Kennedy, whose personal motives and approaches played a pivotal role in the planning and execution of the invasion. Bissell's individualism in particular stemmed from a desire for power and respect, whereas Kennedy's revolved around control.~\parencite{JeffreysJones2003}

    Richard Bissell became the CIA's Director of Plans on January 1, 1959, and just over a year later he proposed a plan to overthrow Fidel Castro.~\parencite{Wyden1979} This bold move by someone so recently appointed raises the question: what motivated it? The answer lies in Bissell's personal goals. Simply put, as Peter Wyden highlights in \textit{Bay of Pigs: The Untold Story}, he wanted to ``go big or go home''.

    \begin{quotation}
        “His attitude was a career gamble… If the operation succeeded, Bissell would be the unquestioned hero of the agencies' most ambitious success.” \parencite{Wyden1979}
    \end{quotation} 

    Bissell sought to secure a legacy of his own, gaining honor and respect in the process, and he viewed Castro's removal as his path to this. This is clearly not howl large scale invasions of foreign countries should be motivated. 

    Bissell's individualism is additionally evident in his decision-making process. For instance, he coordinated an offer to the Mafia, promising \$150,000 for Castro's assassination—a decision made by him and one other person.~\parencite{Wyden1979} Allen Dulles, the CIA Director at the time, was not briefed on this assassination bribe until over six months later~\parencite{Wyden1979}. This highlights the fragmented nature of the CIA, which often operated as isolated factions rather than a unified organization. Hiring two members of the Mafia, listed among the FBI's “top ten most-wanted criminals,” underscores this. According to Wyden\ldots

    \begin{quotation}
        “Bissell knew who he was dealing with… On October 18th, he received a memo from the FBI.”~\parencite{Wyden1979}
    \end{quotation}  

    However, despite knowing that the FBI was actively pursuing the mobsters, he still chose to hire them. Actions like these exemplify the rivalry between the government agencies, as CIA decisions were reportedly influenced by\ldots

    \begin{quotation} 
        “bureaucratic rivalry and the relative prestige of rival intelligence organizations, notably the military intelligence agencies and the FBI.”~\parencite{JeffreysJones2003}.
    \end{quotation} 
    
    In this case, the CIA actively worked with individuals the FBI was trying to apprehend—a clear conflict of interest and an example of a splintered government.

    Furthermore, The CIA and Kennedy were at odds throughout the planning process, clashing over whether the invasion should proceed and how it should be executed. When Kennedy proposed changes to the plan that the CIA knew would compromise the operation, they chose not to correct him or mention the flaws. Their strategy was to \ldots
    
    \begin{quotation}
        ``consciously allow Kennedy to ignore the central weaknesses of the invasion plan… hoping to steer past him a project he deeply mistrusted, but they nevertheless wished to carry out.'' \parencite{Vandenbroucke1984} 
    \end{quotation}
    
    This quote shows how when the parties had conflicting goals the CIA chose deceit over negation, creating a further divide. This deception exemplifies the fact that policymakers chose to act largely in their own interests.

    The culture of individualistic policy was not limited to the CIA: it extended to the executive branch as well. 

    \begin{quote}
        ``Policymakers have ignored CIA estimates, though knowing them to be correct, in pursuit of what they deem to be higher objectives.''~\parencite{JeffreysJones2003}.
    \end{quote}
    
    President Kennedy embodied this. Upon taking office on January 20, 1961, Kennedy sought to assert his authority and reorganize the central government. As noted in \textit{The CIA and American Democracy}, 

    \begin{quotation}
        ``Like any new president, he had to impose his personal authority… It is possible that Kennedy, in some ways an untried and unproven young man, felt this need psychologically to a greater extent.''~\parencite{JeffreysJones2003}
    \end{quotation}  

    This shows Kennedy's approach to governance was deeply personal, reflecting his desire to maintain hands-on control over policy. Kennedy wished to have a more fine degree of control than previous presidents.

    Kennedy's relationship with the CIA underscores this point. He was a vocal critic of the agency and notoriously cut funding for covert operations—a move described as 
    
    \begin{quotation}
        ``consistent with Kennedy's desire to exert greater personal influence over foreign-policy formation.''~\parencite{JeffreysJones2003}
    \end{quotation}

    This decision was driven by his want for centralized control rather than rational policy, as he hoped that these cuts would force the CIA to be more public.

    The individualistic nature of the operation leaders meant the actual plans and execution were also splintered. Planners were totally unaware that the Bay of Pigs had transformed into Playa Girón—a Cuban equivalent of Coney Island—in the three years leading up to the invasion. \parencite{Wyden1979} Troops received ammunition for weapons they did not have and endured hours of airstrikes before landing, only to face Cuban defenses head-on. The result was catastrophic, clearly showing the problem with the individualistic nature of the planning, as the planners were more concerned with their personal goal in result of the invasion rather than actually planning it. 
    
    This failed operation had long term repercussions, because, in the eyes of the world, it solidified Fidel Castro's argument. According to an article on \textit{history.state.gov}, 

    \begin{quotation}
        ``The failed invasion strengthened the position of Castro's administration, which proceeded to openly proclaim its intention to adopt socialism and pursue closer ties with the Soviet Union.'' \parencite{state_bayofpigs}
    \end{quotation}

    This is undoubtedly true and specifically the opposite of the U.S. goal of the invasion. Additionally, the invasion served as of one of the first distinct markers in the ruin of U.S. Cuban relations, setting a trend of animosity which still remains to this day. Furthermore, it can be argued that the grand failure of the Bay of Pigs lead to some of the other major events in world polities, with one of the most notable being the Cuban Missile Crisis. \parencite{cfr_uscuba} 

    In conclusion, the Bay of Pigs was a disaster due to unchecked self-interested authority figures in the government. These individualistic policies lead to the entire operation having an outcome which was directly at odds with its own goals with the self-interested goals of a small group individuals directly affected politics on the world stage.

    \textit{Full disclosure, I have discussed this a previous paper for another class, so some of the research has been reused for this.}

\section{Iraq}

    \subsection{Redeployment}
        In \textit{Redeployment} by Phil Klay \parencite{Klay2014}, a \textbf{fictional}\footnote{\textbf{Disclaimer:} While this is a fictional story, it is written by a veteran, which gives it an amount of merit. That being said, as it is fiction, some aspects are likely to be hyperbolic.} 
        story written a veteran from Iraq, we see a collection of short stories based off of his deployment. One of them, \textit{Money as a Weapons System}, describes someone's experience working for a USAID program in Iraq. He quickly finds himself caught up in frivolous spending, as he finds that certain people have outrageously high wages and that programs are being run with no forcible outcome, all while that he is forced to impose the uninformed will of senators.

        Our character is at the mercy of fictional U.S. businessman named Gene Goodwin, who is described as ``the mattress king of north Kansas''. Due to his wealth he has a large amount of influence over senators, and by extension, U.S. involvement in Iraq. He uses this influence to push his own ideas, all of which are deeply flawed. One example his flawed ideas is that of teaching Iraqi children how to play baseball. In order to do this he sends shipments of baseballs, bats, and uniforms to our main character. Our main character, as well as everyone except Goodwin himself, realize how nonsensical this idea is, leading to our narrator engaging in this conversation\ldots
        
        \begin{quotation}
            “\ldots It's all the rage at the embassy, he said. It's been very effective.” \\
            “Very effective at what?” \\
            “Well,” said the major, beaming, “I'm not sure, but they make for some great photos.” \\
            I took a deep breath. “Chris Roper thinks this is a good idea?” \\
            “Absolutely not,” said the major, an expression of outrage on his face. \\
            “Then Representative Gordon…,” I said. \\
            “I don't think so,” said Major Zima. “But he did tell me and the colonel what a key constituent Mr. Goodwin was, and how angry Mr. Goodwin was that no one seemed to take his baseball plan seriously.” \\
            “And you told him the ePRT guys could handle it.” \\
            “I said you'd be honored.” \parencite{Klay2014}
        \end{quotation}

        In this quote our main character is instructed to work on the baseball project which is ``all the rage'' at the embassy. The major instructing him to work on the project doesn't see any validity in the idea, nor does his superior. However, because Mr. Goodwin (the wealthy businessman) is pushing the idea, he is forced to proceed anyway. This is a fictional example of very real cases regarding nonsensical waste and projects that is pushed due to one individual. These individual generally don't have sort of any formal understanding and yet are able to push their own agenda. The nature of this is incredibly dangerous, especially when it comes to program decisions like this one. 
        
    \subsection{Commission on Wartime Contracting (CWC) Report}
        In Jul of, 2007, Senators Jim Webb and Claire McCaskill introduced a bill to form the independent and bipartisan commission~\cite{CWC_2011}. The CWC was founded in order to find areas with waste and fraud and then provide recommendations to congress on how to improve. The commission provided five reports, however, the one discussed here is the final of them. This report is one of the best documentations of excess spending and waste in Iraq. The CWC found that at least \$31 billion, with a possibility of up to \$60 billion, was lost to waste a and fraud.\footnote{\textbf{Note:} This commission is likely to have a large amount of bias due to the nature of its funding, and thus this value is likely to be an underestimation.} 
        This waste occurred in a couple of key sectors, which the largest one being overzealous contracts with PMC groups and fraud within them, but also, to a lesser extent, within USAID programs. However, it is not just a matter of wasted taxpayer dollars, as the CWC argues that lives were also lost. 
        
        The CWC was incredibly critical of the military, exposing just how truly rampant this waste was. During the final hearing of the CWC, Claire McCaskill said that 
        
        \begin{quote}
            ``I have taken trips to Iraq and Afghanistan, where I have seen with my own eyes the lack of planning, inadequate oversight, and sheer waste in our contingency contracting operations. I can tell a number of anecdotal stories about my visits to both Iraq and Afghanistan on contracting oversight trips. But I particularly remember the time when I asked a general in Kuwait, where a lot of the contracting work was done, `how did this happen? How did this get so out of control?''~\parencite{us_senate2011wartime_contracting}
        \end{quote}

        This is after a general\footnote{\textbf{Note:} the name of this general is not disclosed} said to her that 

        \begin{quote}
            ``I wanted three kinds of ice cream in the mess hall yesterday, and I didn't care what it cost.''~\parencite{us_senate2011wartime_contracting}
        \end{quote}

        However, the issue here is \textit{not} the spending, it's the fact that one general had the individual power to do that. One general should not be able to have the power to be able to spend, or enact any large-scale plans, without proper oversight. When certain individuals can push their own idea and agenda without worry, like this general, it leads to disastrous results. The fundamental idea of Democracy, and really any efficient power system, relies on a system of checks and balances. If certain officials can make decisions in their own personal interests, and that isn't recognized and stopped, it inevitably leads to issues. Witness senator Kelly Ayotte echoes this when quoting general Petraeus\ldots

        \begin{quote}
            ``If\ldots we spend large quantities of international contracting funds quickly and with insufficient oversight, it is likely that some of those funds will unintentionally fuel corruption, finance insurgent organizations, strengthen criminal patronage networks, and undermine our efforts in Afghanistan.'' \parencite{us_senate2011wartime_contracting}
        \end{quote}

        This is \textit{exactly} what happened, as stated by the BBC, it was found that Hazem Shalaan, the Iraqi prime minister of defense, did just that. The BBC state that\ldots

        \begin{quote}
            ``He and his associates siphoned an estimated \$1.2bn out of the ministry. They bought old military equipment from Poland but claimed for top-class weapons. Meanwhile[,] they diverted money into their own accounts.'' \parencite{bbc2008iraqcontracts}
        \end{quote}

        These contracts given to military officials in Iraq, with the intention of having them be used for actual defense equipment, were instead used for the officials own personal gain. The unchecked stream of funding to Shalaan lead to it being used in a way which directly hurt U.S. goals as well as furthering corruption in Iraq.  
        
        This behavior mirrors our ice cream general, however, on a much more serious scale. It further shows that in cases where officials act in their own interests, it is likely to lead to a compromise in the operation. This individualistic policy lies rooted in U.S. officials and compromises large scale actions in foreign policy.
        
        % \textit{Note: I plan to add a lot more here regarding the CWC report and the hearing, however, due to the very lengthy nature of both, I have not had the time to fully comb through them yet.}

    \subsection{Halliburton, Dick Cheney, and a hell of a lot of oil}
        Corruption is ever-present in government, but in times of war, it becomes rampant. However, Widespread corruption is not the focus here, instead it is how \textit{one} person's corruption impacted Iraq. Dick Cheney and his desire for oil profiting in Iraq is a prominent example of this. From 1995 to 2000 Dick Cheney was the CEO of Halliburton, a large oil company. Before this he was the secretary of defense from 1989 to 1993 and afterward was Vice President from 2001 to 2009. This tie is already suspicious, but it quickly becomes incredibly egregious. 
        
        Halliburton had a very involved role in Iraq after winning a very large government contract. This contract was incredibly lucrative, drawing in an estimated~\$10 billion in profit \parencite{nytimes2005halliburtoncontract}. A very important aspect is this contract was \textit{non-competitive}, meaning that only Halliburton was the only company which had the opportunity to bid on it. \parencite{bbc2008iraqcontracts} This is very likely due to internal favoritism, and with the former CEO as VP,\footnote{\textbf{Disclaimer:} Due to the heavily politicized nature of this story, there are a lot of sources from both the left and the right saying widely different things (ex. Cheney is still on the payroll, Cheney isn't, Cheney's still making money off of his stocks, Cheney's stocks are going to charity, etc.) The truth is most likely somewhere in the middle and I have tried to mirror that in my analysis.}
        it seems clear where that comes from. 
        
        The egregious nature of this contract is further shown by the fact that the chief overseer of contracts at the Army Corps of Engineers, Bunnatine H. Greenhouse, was fired from her position after saying that the contract was\ldots

        \begin{quote}
            ``[T]he most blatant and improper contract abuse I have witnessed during the course of my professional career.~\parencite{nytimes2005halliburtoncontract}''
        \end{quote}  

        This comment being made by the chief overseer, along with her prompt firing thereafter, truly shows just how bad the nature of this contract favoritism was. 

        However, there is a greater problem here then just the waste. Because Cheney had an invested interest in Halliburton, there is a massive conflict of interest. Cheney's interests thus no longer fully fall into what is beneficial for the U.S. or Iraq, but instead is also split between his stake in Halliburton. This is a problem because as discussed previously, policy which is likely based off of the interests of a single individual will inevitably create further issues.
        
        % \textit{Note: This feels mostly done, needs some polishing, and will probably add another example and some more quotes, however, there is not a lot of \textbf{unbiased} coverage.}
        
     \subsection{Long Term Effects}   
        You could ask pretty much any American or Iraqi about their opinion on U.S. involvement in Iraq and hear a similar story, one about a war which served little purpose. Very few Americans look back on our involvement and think of it highly. Instead, it is generally regarded as a mistake, something that the U.S. could, and should have, avoided. \parencite{igielnik2019wars} 

        The war in Iraq has had drastic long term effects on the region, with these flawed contracts leading to a large increase in the militarization of not just Iraq, but the world as a whole. According to the Atlantic Council,

        \begin{quotation}
            ``When the United States invaded Iraq two decades ago, one of the public justifications for the war was that it would help spread democracy throughout the Middle East. The invasion, of course, had the opposite effect: it unleashed a bloody sectarian conflict in Iraq, badly undermining the reputation of democracy in the region and America's credibility in promoting it.'' \parencite{atlanticcouncil_iraqwar2023}
        \end{quotation}

        Furthermore

        \begin{quotation}
            ``\ldots the consequences of the war led to the spectacular empowerment of armed nonstate actors in the region and beyond, who launched a full-frontal assault on the sovereignty of many states. The Islamic State of Iraq and al-Sham, of course, emerged amid the brutal contestation of power in post-invasion Iraq and pursued its “caliphate” as an alternative (Sunni) political institution to rival the nation-state. While the threat has been contained \ldots it is only beginning to gather force on the African continent. In addition, because Iran effectively won the war in Iraq, it was able to sponsor a deep bench of Shia nonstate groups which have eroded state sovereignty in Lebanon, Syria, Yemen, and Iraq itself.'' \parencite{atlanticcouncil_iraqwar2023}
        \end{quotation}

        This shows that the consequences of Iraq included further destabilization of the region, which was what the US sought to avoid in the first place. 

        While it is not possible to fully conclude that individualism directly lead to the failure of U.S. involvement in Iraq, it is undoubtedly true that it heavily contributed to it. These individuals and their personally motivated interests had a much greater impact then they should have on the entire war, and by extension, its results on the world. This very similar to the Bay of Pigs, where the operation failed due to internal issues, the majority of which were brought upon by individuals.  

\section{Conclusion}
        In conclusion, the individualistic policy of U.S. officials has directly caused adverse results. We notably see this in the Bay of Pigs invasion, which failed due to individualistic policy and result in the strengthening of the Castro regime. We further see this in the U.S. invasion of Iraq, where self-interested individuals served to undermine U.S. involvement, leading to furthered instability in the region, as well as increased militarization. These individuals have played large parts in both the worlds' perception of the U.S. and the current political climate of the world. Lastly, these individual have had a profound impact on shaping the state of the modern day world.

\pagebreak

\defbibheading{bibintoc}{%
  \section*{\centering Works Cited}
  \addcontentsline{toc}{section}{Works Cited}
}
\printbibliography[heading=bibintoc]

\end{document}