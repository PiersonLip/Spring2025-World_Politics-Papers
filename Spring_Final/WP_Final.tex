\documentclass{article}
\usepackage[
  backend=biber,
  sorting=none,
  style=mla
]{biblatex}
\usepackage{setspace}
\usepackage{hyperref}
\sloppy
\title{The US and Rampant Individualism }
\author{Pierson Lipschultz}

\doublespace
\addbibresource{sources.bib}

\begin{document}
\maketitle

% indenfity if it is unkown quote 
% why these cases
    % national security archvie
    % mla or apa
% redployment first
\begin{abstract}
    In this paper I investigate the question ``How have certain US political figures and their personal agendas affected world politics?''
\end{abstract}


\section{Introduction}
    The United States has had, ever since its very inception, a deep problem of individuals having a much greater influence then they should politically, both at home and abroad. We have seen this is an enormous amount of conflicts and decisions, but some of the most key examples of this are the Bay of Pigs, and more recently, the waste during the war in Iraq.\footnote{I feel like it is very important to make a clear distinction here. As I did research for this paper, a recurrent  theme I found was recently appointed government officials talking about \textit{``waste''} in USAID, but their claims very quickly fall apart when placed under scrutiny. A key example of this being the ``\$20 million for Sesame Street'', which does not hold up.} 

    However, there is a deeper issue here then just waste. If a politician has a conflict of interest and invested stake in certain outcome (say, for example, the war continuing because they are making money off of it), their policy will mirror that. This means that their policies, which have drastic impacts on both the US and the world, is effectively compromised, as they are no longer acting in the interest of their nation.

\section{Bay of Pigs}

    The Bay of Pigs is undoubtedly one of the most public and embarrassing moments in the history of the CIA.\@ It's a disaster so large people look at it and wonder how it could have even happened, but in actuality is quite simple. It was due to the individualistic nature of policy, both within the organization and broader government. This ideology is evident in the actions of key figures such as Richard Bissell and John F. Kennedy, whose personal motives and approaches played a pivotal role in the planning and execution of said invasion. Bissell's individualism stemmed from a desire for power and respect, while Kennedy's revolved around control. Examples of this dynamic are also present in the actions of high-ranking CIA officials and Nixon's influence.~\parencite{JeffreysJones2003}

    Richard Bissell became the CIA's Director of Plans on January 1, 1959, and just over a year later, he proposed a plan to overthrow Fidel Castro to President Eisenhower~\parencite{Wyden1979}. This bold move by someone so recently appointed raises the question: what motivated it? The answer lies in Bissell's individualistic goals. Simply put, he wanted to “go big or go home.” As Peter Wyden highlights in \textit{Bay of Pigs: The Untold Story}, 
    \begin{quotation}
        “His attitude was a career gamble… If the operation succeeded, Bissell would be the unquestioned hero of the agencies' most ambitious success.”
    \end{quotation} 
    Bissell sought to secure a legacy of his own, gaining honor and respect in the process, and he viewed Castro's removal as his path to this, taking an approach which was all or nothing.

    Bissell's individualism is additionally evident in his decision-making process. For instance, he coordinated an offer to the Mafia, promising \$150,000 for Castro's assassination—a decision made by him and one other person~\parencite{Wyden1979}. Allen Dulles, the CIA Director at the time, was not briefed on the assassination bribe until over six months later~\parencite{Wyden1979}. This highlights the fragmented nature of the CIA, which often operated as isolated factions rather than a unified organization. Hiring two members of the Mafia, listed among the FBI's “top ten most-wanted criminals,” underscores this. According to Wyden, \begin{quotation}
        “Bissell knew who he was dealing with… On October 18th, he received a memo from the FBI.”~\parencite{Wyden1979}
    \end{quotation}  
    However, despite knowing that the FBI was actively pursuing the mobsters, he still chose to hire them. Actions like these exemplify the rivalry between government agencies, as CIA decisions were reportedly influenced by 
    \begin{quotation} 
        “bureaucratic rivalry and the relative prestige of rival intelligence organizations, notably the military intelligence agencies and the FBI.”~\parencite{JeffreysJones2003}.
    \end{quotation} 
    In this case, the CIA actively worked with individuals the FBI was trying to apprehend—a clear conflict of interest and an example of a splintered government.

    The Bay of Pigs invasion illustrates the disastrous consequences of such individualistic policies. The CIA and Kennedy were at odds throughout the planning process, clashing over whether the invasion should proceed and how it should be executed. When Kennedy proposed changes to the plan that the CIA knew would compromise the operation, they chose not to correct him or mention the flaws. Their strategy was to \ldots
    
    \begin{quotation}
        ``consciously allow Kennedy to ignore the central weaknesses of the invasion plan… hoping to steer past him a project he deeply mistrusted, but they nevertheless wished to carry out'' \parencite{Vandenbroucke1984} 
    \end{quotation}
    
    This quote shows how both parties had conflicting goals and how the CIA chose deceit over negation, creating a further divide. This deception exemplifies the fact that policymakers chose to act largely in their own interests.

    The culture of individualistic policy was not limited to the CIA; it extended to the executive branch as well. 

    \begin{quote}
        ``Policymakers have ignored CIA estimates, though knowing them to be correct, in pursuit of what they deem to be higher objectives.''~\parencite{JeffreysJones2003}.
    \end{quote}
    
    President Kennedy exemplified this. After taking office on January 20, 1961, Kennedy sought to assert his authority and reorganize the central government. As noted in The CIA and American Democracy, 

    \begin{quotation}
        ``Like any new president, he had to impose his personal authority… It is possible that Kennedy, in some ways an untried and unproven young man, felt this need psychologically to a greater extent.''~\parencite{JeffreysJones2003}
    \end{quotation}  

    This shows Kennedy's approach to governance was deeply personal, reflecting his desire to maintain hands-on control over policy. Kennedy wished to have a more fine degree of control than previous presidents.

    Kennedy's relationship with the CIA underscores this point. He was a vocal critic of the agency and notoriously cut funding for covert operations—a move described as 
    
    \begin{quotation}
        ``consistent with Kennedy's desire to exert greater personal influence over foreign-policy formation.''~\parencite{JeffreysJones2003}
    \end{quotation}

    This decision was driven by his want for centralized control rather than by true policy, as he hoped that these cuts would force the CIA to be more public.

    The individualistic nature of the operation leaders meant the actual plans and execution were also splintered. Planners were totally unaware that the Bay of Pigs had transformed into Playa Girón—a Cuban equivalent of Coney Island—in the three years leading up to the invasion~\parencite{Wyden1979}. Troops received ammunition for weapons they did not have and endured hours of airstrikes before landing, only to face Cuban defenses head-on. The result was catastrophic and clearly showed the problem with the individualistic nature of the planning, as the planners were more concerned with the personal effects of the invasions than actually planning it. 

    This lead to the entire operation being a complete disaster, with all the work and planning going into it be a complete waste. This operation had long term repercussions, because, in the eyes of the world, it solidified Fidel's argument.

    \textit{Full disclosure I have discussed this a paper from a previous class, so some of the research has been reused for this.}

\section{Iraq}

    \subsection{Redeployment}
        In \textit{Redeployment} by Phil Klay \parencite{Klay2014}, a \textbf{fictional}
        \footnote{\textbf{Disclaimer:} While this is a fictional story, it is written by a veteran, which I think gives it a relative amount of merit. Unfortunately, it is incredibly hard to find actual specific accounts from Iraq, most likely due to the recent nature of it and the lack of declassified files. That being said, as it is fiction, most likely is quite hyperbolic} 
        story written a veteran from Iraq, we see a collection of short stories based off of his deployment. One of them, \textit{Money as a Weapons System} describes someone's experience working for USAID in Iraq. He finds that he is part of frivolous programs and is forced to  go through various contracts due to senators or other government officials who do not truly know what is going on the ground. 

        Our main character finds that certain people have outrageously high wages, that programs are being run with no forcible conclusion, and that he is forced to impose the will of senators who have true grasp of what's going on. 

        Our character is at the mercy of fictional US businessman named Gene Goodwin, who is described as ``the mattress king of north Kansas''. Due to his wealth he has a large amount of influence over senators, and by extension, US involvement in Iraq. He uses this influence to push his own ideas, all of which are deeply flawed. One example of this is his idea of teaching Iraqi children how to play baseball, to do this, he sends shipments of baseballs, bats, and uniforms to our main character and various bases in Iraq. Our main character, as well as everyone involved, fully understand how nonsensical this idea is. With our the narrator engaging in this conversation\ldots
        
        \begin{quotation}
            “\ldots It's all the rage at the embassy, he said. It's been very effective.” \\
            “Very effective at what?” \\
            “Well,” said the major, beaming, “I'm not sure, but they make for some great photos.” \\
            I took a deep breath. “Chris Roper thinks this is a good idea?” \\
            “Absolutely not,” said the major, an expression of outrage on his face. \\
            “Then Representative Gordon…,” I said. \\
            “I don't think so,” said Major Zima. “But he did tell me and the colonel what a key constituent Mr. Goodwin was, and how angry Mr. Goodwin was that no one seemed to take his baseball plan seriously.” \\
            “And you told him the ePRT guys could handle it.” \\
            “I said you'd be honored.” \\
        \end{quotation}

        This story of a people being forced to work on projects they know are doomed to failed is ever-present in the history of the US, with Iraq being a prime example of this. 
        
    \subsection{Commission on Wartime Contracting (CWC) Report}
        In 2008 congress created the independent and bipartisan Commission on Wartime Contracting in Iraq and Afghanistan~\parencite{CWC_2011}. The CWC was founded in order to find places with excessive waste and fraud and to provide recommendations to congress on how to improve. This commission provided five reports, however, the one detailed here is the final of them. This report is one of the better documentations of excess spending and waste in Iraq. They CWC found that at least \$31 billion, with a possibility of up to \$60 billion, was lost to waste a and fraud.\footnote{It is important to note that this commission will have a large amount of bias due to the nature of its funding, and thus this value is likely to be an underestimation} 
        The waste was in a couple of key sectors, which the largest one being overzealous contracts with PMC groups and fraud within them, but also, to a lesser extent, within USAID programs.

        However, it is not just taxpayer dollars which are lost, the CWC argues that lives were also lost. 
        
        The CWC was incredibly critical of the government, exposing just how truly rampant the waste was. During the final hearing of the CWC, Claire McCaskill said that 
        
        \begin{quote}
            ``I have taken trips to Iraq and Afghanistan, where I have seen with my own eyes the lack of planning, inadequate oversight, and sheer waste in our contingency contracting operations. I can tell a number of anecdotal stories about my visits to both Iraq and Afghanistan on contracting oversight trips. But I particularly remember the time when I asked a general in Kuwait, where a lot of the contracting work was done, `how did this happen? How did this get so out of control?''~\parencite{us_senate2011wartime_contracting}
        \end{quote}

        This is after a general\footnote{Note that the name of this general is not disclosed} said to her that 

        \begin{quote}
            ``I wanted three kinds of ice cream in the mess hall yesterday, and I didn't care what it cost.''~\parencite{us_senate2011wartime_contracting}
        \end{quote}

        The issue here is \textit{not} the spending, it's the fact that one general had the individual power to do that. 
        When the witness senator Kelly Ayotte was called up to the stand, she quoted general Petraeus stating that 

        \begin{quote}
            ``If\ldots we spend large quantities of international contracting funds quickly and with insufficient oversight, it is likely  that some of those funds will unintentionally fuel corruption, finance insurgent organizations, strengthen criminal patronage networks, and undermine our efforts in Afghanistan.'' \parencite{us_senate2011wartime_contracting}
        \end{quote}

        This is \textit{exactly} what happened, as stated by the BBC, it was found that Hazem Shalaan, the Iraqi prime minister of defense, did just that. The BBC state that

        \begin{quote}
            ``He and his associates siphoned an estimated \$1.2bn out of the ministry. They bought old military equipment from Poland but claimed for top-class weapons. Meanwhile[,] they diverted money into their own accounts.'' \parencite{bbc2008iraqcontracts}
        \end{quote}

        This goes to show that misscordinated funding leads to larger issues than just lost taxpayer dollars, instead also undermining the very goal of the US's involvement in Iraq. This individualistic
        
        \textit{Note: I plan to add a lot more here regarding the CWC report and the hearing, however, due to the very lengthy nature of both, I have not had the time to fully comb through them yet.}

    \subsection{Halliburton, Dick Cheney, and a hell of a lot of oil}
        In Iraq a lot of government officials had direct and key ties to oil companies, with politicians receiving large amount of funding from these companies for certain policy decisions.

        Rampant corruption is ever-present in government, but in times of war, it becomes rampant. Corruption is not the focus here, it is how \textit{one} person's corruption, Cheney's, impacted Iraq. Cheney's love for oil in Iraq is a great example of this.

        From 1995 to 2000 Dick Cheney was the CEO of Halliburton, a large oil company. Before this he was the secretary of defense from 1989 to 1993 and afterward was Vice President from 2001 to 2009. This tie already would raise some eyebrows, but it quickly becomes incredibly egregious. 
        
        Halliburton had a very involved role in Iraq, as they won a government contract. This contract was incredibly lucrative, drawing in an estimated~\$10 billion. A very important aspect is this contract was \textit{non-competitive}, meaning only Halliburton got to bid on it \parencite{bbc2008iraqcontracts}. This is very clearly some sort of internal favoritism, and with the former CEO as VP,\footnote{Import disclaimer here. As with more recent stories, there are a lot of sources from both the left and the right saying widely different things (Cheney is still on the payroll, Cheney 
        isn't, Cheney's still making money off of his stocks, Cheney's stocks are going to charity, etc. The truth is, of course, somewhere in the middle.)}
        it seems clear where that comes from. The chief overseer of contracts at the Army Corps of Engineers, Bunnatine H. Greenhouse, was fired from her position after saying that the contract was 

        \begin{quote}
            ``[T]he most blatant and improper contract abuse I have witnessed during the course of my professional career.~\parencite{nytimes2005halliburtoncontract}''
        \end{quote}  
        
        Contracts like these are a key aspect of waste in government. Greenhouse states that they discovered Halliburton to be actively charging artificially higher rates for gas to soldiers.~\parencite{nytimes2005halliburtoncontract} However, there is a greater problem here then just the waste, as if politicians are acting in a way which is knowingly wasteful, in ways which benefit themselves, their policy is, in a sense, compromised. 

        \textit{Note: This feels mostly done, needs some polishing, and will probably add another example and some more quotes, however, there is not a lot of \textbf{unbiased} coverage.}
        
        
\section{Trump, Elon, and friends}
    Individualism has always posed problems when it comes to US policy, however, it has generally been happening in the background, away from public view. However, the current administration is an extreme example of this, with politicians opening stating their personal goals and using that as reasoning for their governmental decisions.

    There is a very good argument to be made that this idea of individualism in US policy is more prevalent than it ever has been before. The Trump administration is full of individual actors, all of who have an incredibly large influence on both the US and the world as a whole. We have seen this with a number of actors, including 
    \begin{itemize}
        \item RFK
        \item Hegseth 
        \item Trump
        \item Elon
    \end{itemize}

    \textit{\textbf{Note:} I realized while writing this it would be very possible to directly tie the interests of certain individuals in the current political scene with this sense of individualistic policy. It would be a very recent tie in. I was planning on looking into Elon and RFK primarily, as Elon has been public about his ideas of what government should be, talking about neorealism in interviews and such. His very personal ideas have a large effect on global politics. I am debating talking about Trump himself as well.}
\pagebreak
\printbibliography[
    heading=bibintoc,
    title={\centering Sources}
    ]

    
\end{document}